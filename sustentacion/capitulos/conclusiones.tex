
	\section{Conclusiones}
		\begin{frame}{Conclusiones}
			En el presente trabajo se estudió las oscilaciones magnéticas en una estructura basada en grafeno con un potencial periódico perturbativo. A partir del enfoque teórico planteado, basado en el formalismo del hamiltoniano de transferencia, se encontró una expresión de la conductividad eléctrica longitudinal de una estructura basada en grafeno como función de un campo magnético aplicado y de la temperatura. Se estudió el fenómeno de oscilaciones magnéticas de la conductividad longitudinal para el caso de altas temperaturas.\\
			De manera general se encontraron los siguientes resultados:

		\end{frame}

		\begin{frame}
		\footnotesize

		\begin{enumerate}
		 	\item 
			La conductividad longitudinal tiene dos contribuciones: una de Boltzmann y otra de origen cuántico. 
			
			\item 
			La contribución de Boltzmann determina el carácter monótono decreciente de la conductividad, en tanto la contribución cuántica determina el comportamiento oscilatorio de la conductividad longitudinal a altas temperaturas.
			
			\item 
			Con el aumento de la temperatura, en promedio el producto $\tau(\epsilon)\epsilon S/\epsilon_{f}$ se mantiene constante para un valor de campo dado. Lo anterior conduce a que el efecto oscilatorio no se observe en la contribución de Boltzmann.
			
			\item 
			Valores enteros del parámetro de desorden, que corresponden a la variaciones enteras del parámetro de Dingle, dan como resultado que el efecto oscilatorio se mantiene a altas temperaturas en la contribución cuántica.
			
			\item 
			Los resultados obtenidos en el presente trabajo pueden explicar de manera cualitativa resutados experimentales de oscilaciones magnéticas a altas temperaturas en estucturas basadas en grafeno \cite{Kumar2017}.
		\end{enumerate}
			
		\end{frame}