\section{Espectro de energía}
\justifying

\subsection{Niveles de Landau}
\begin{frame}{Niveles de Landau}
	\begin{multicols}{2}
	En presencia de un campo magn\'etico, los electrones solo pueden ocupar \'orbitas con estados discretos de energ\'ia,
	llamados niveles de Landau (L.L. Por sus siglas en ingl\'es):
	\begin{equation}
		E_n = \hbar \omega_c (n+\frac{1}{2})
		\label{eq:landau}
	\end{equation}
	donde  $\omega_c=\frac{eB}{mc}$ es la frecuencia del ciclotrón.

	\begin{figure}[b!]
		\includegraphics[width=3cm,height=4.5cm]{graficas/LL_material.png}
			\caption{\scriptsize{Niveles de Landau en material convencional}}
	\end{figure}
	\end{multicols}
\end{frame}

\begin{frame}{Niveles de Landau en el grafeno}
	\begin{multicols}{2}
		\scriptsize{Los niveles de Landau se pueden obtener a partir de considerar el hamiltoniano,
			$\hat{H}=\frac{\hat{\pi}^2}{2m}+V(\vec{r})$ siendo $\hat{\pi}=\vec{p}- \frac{e}{c}\vec{A}$, de la forma:
			\begin{equation}
				\hat{H} = \sqrt{\frac{2e\hbar B \nu_f}{c}}
				\left( \begin{array}{c c}
					0&\hat{a}\\
					\hat{a}^\dagger&0
				\end{array} \right)
			\end{equation}
			donde $\hat{a}= \sqrt{c/2e\hbar B}(\pi_x-i\pi_y)$ y\\	$\hat{a}^\dagger=\sqrt{c/2e\hbar B}(\pi_x+i\pi_y)$.\\
			Resolviendo la ecuación de Schrödinger se obtiene:
			\begin{equation}
					E_n= \pm \hbar\omega_c \sqrt{n}
			\end{equation}
			donde $\omega_c = \sqrt{2}\nu_f/l_B$ y $l_B = \sqrt{\frac{\hbar c}{e B}}$.}
		\begin{figure}
			\includegraphics[height=4cm]{graficas/LL_grafeno.png}
			\caption{\scriptsize{Niveles de Landau en el grafeno}}
		\end{figure}
	\end{multicols}
\end{frame}

\subsection{Estados de Bloch en grafeno en un campo eléctrico}

\begin{frame}{Estados de Bloch en el grafeno}
	Para un sistema sustrato-grafeno, el hamiltoniano se puede escribir de la forma:
	\begin{equation}
			\hat{H} = \hat{\vec{p}}+\frac{e}{c}\hat{\vec{A}} + \hat{U},
	\end{equation}

	\begin{figure}[t]
		\centering
		\includegraphics[scale=0.4]{graficas/Moire.jpg}
		\caption{Patrón de Moiré en una capa de grafeno sobre hBN.}
		\label{moire}
	\end{figure}
\end{frame}

\begin{frame}
	donde $\hat{U}(x,y)$ puede ser aproximado por la expresión \cite{Wang2004}:
	\begin{equation}
			\hat{U}(x,y)=U_{x}\cos (K_{x}x)\cos (K_{y}y)+\frac{U_{y}}{2}\left[ 1+\cos (2K_{y}y)\right],
	\end{equation}
	siendo
	\begin{equation}
		K_{x}=\frac{2\pi}{a_{x}},\hspace{1cm}K_{y}=\frac{2\pi}{\sqrt{3}a_{y}},
	\end{equation}
	donde $a_x$ y $a_y$ son los periodos de la celda elemental.
\end{frame}

\begin{frame}
	El espectro de energía está dado por la expresión \cite{Wang2004}:
	\begin{equation}
			\epsilon = \pm \epsilon_n + \zeta(k_x,k_y),
	\end{equation}
	donde $\epsilon_n = \hbar\omega_f\sqrt n $ y
	$\zeta(k_x,k_y)$ es la energía asociada a la estructura de estados de Bloch.
	Por lo tanto, las bandas de Landau se dividen en subbandas,
	lo que lleva a una ampliación del espectro magnético.
\end{frame}

\begin{frame}
	Dicha energía puede expresarse de la siguiente manera \cite{Wang2004}:
	\begin{equation}
			\zeta(k_{y}, \beta)=\frac{U_{y}}{2}\left[1+F_{n}(4u_{y})\cos 2\beta\right]+U_{x}F_{n}(4u_{y})\cos \gamma\cos \beta,
	\end{equation}
	donde
	\begin{equation}
		F_{n} (4u_{y}) = e^{- u_{y} / 2} \left[L_{n} (4u_{y}) + L_{n-1} (4u_{y}) \right],
	\end{equation}

	$L_n$ son los polinomios de Laguerre y $ u_{y} = l^{2} K_{y}^{2} / 2 $
\end{frame}

\begin{frame}
	\begin{equation}
		\gamma = K_{x} l^{2} (k_{y} + K_{y}) 2,\hspace{1cm} \beta = l^{2} K_{y} k_{x }
	\end{equation}

	Las componentes $ k_{y} $ y $ \beta $ se definen dentro de los intervalos
	\begin{equation}
	-a_{x} / 2l^{2} \leq k_{y} \leq a_{x} / 2l^{ 2},\hspace{1cm}0 \leq \beta \leq 2 \pi.                                                                               \end{equation}

  El ancho de las bandas de Landau está modulado por los polinomios de Laguerre $L_n$,
	los cuales son funciones oscilantes de la razón $\Phi_o/\Phi$.
\end{frame}
