\section[Oscilaciones]{Oscilaciones magnéticas}
\justifying

\subsection{Conductividad longitudinal en una capa de grafeno}

\begin{frame}
  La geometría experimental consiste en una barra de Hall de grafeno
  sobre un sustrato de hBN.\\

  \begin{figure}[h]
      \centering
      \includegraphics[scale=0.4]{graficas/hall_bar.png}
      \caption{Imágenes ópticas de barra de Hall multiterminal usadas por Kumar et Al. \cite{Kumar2017}}
      \label{Hall_bar}
  \end{figure}
\end{frame}

\begin{frame}
  Para obtener una expresión de la conductividad,
  se usará el enfoque del hamiltoniano de transferencia \cite{Illera2015}.\\
  La tasa de transición de estados se calcula usando la regla de oro de Fermi \cite{Illera2015}-\cite{Sakurai2017}:

  \begin{equation}\label{golden rule}
    d^{2}W_{l\rightarrow r \in D}=\sum_{\eta'}d^{2}W_{\eta,\eta'}=\frac{2\pi}{\hbar}\sum_{\eta'}|t_{\eta,\eta'}|^{2}\rho_{r}(\epsilon')\delta(\epsilon'-\epsilon)d\epsilon',
  \end{equation}

  $t_{\eta,\eta'}$ representa la transición entre los estados $|\eta|$ y $|\eta'|$ y $\rho(\epsilon')$ es la densidad de estados de la derecha.
\end{frame}

\begin{frame}
  La tasa de transición de los estados ocupados de la izquierda a los no ocupados de la derecha
  en el camino entre los dos contactos de la siguiente forma:
  \scriptsize
  \begin{equation}
    \Gamma_{l\rightarrow r}=\frac{2\pi}{\hbar}\sum_{\eta}\int \rho_{l}(\epsilon)f_{l}(\epsilon)\left\lbrace \int\sum_{\eta''}|t_{\eta',\eta''}|^{2}\rho_{r}(\epsilon'')[1-f_{r}(\epsilon'')]\delta(\epsilon''-\epsilon')d\epsilon''\right\rbrace d\epsilon.
  \end{equation}
  \small
  La corriente de transferencia es:
  \begin{equation}
    I=e(\Gamma_{l\rightarrow r}-\Gamma_{r\rightarrow l}).
  \end{equation}
\end{frame}

\begin{frame}
  Considerando simetria en la matriz de salto ($t_{\eta,\eta'} = t_{\eta',\eta}$),
  \begin{equation}
    I=-g\frac{2\pi e}{\hbar}\sum_{\eta'}\int |t_{\eta,\eta'}|^{2}[f(\epsilon)-f(\epsilon-eU_{g})]\textbf{Im}G_{l}^{R}(\epsilon)\mathbf{Im} G_{r}^{R}(\epsilon-eU_{g})d\epsilon,
  \end{equation}
  donde $f$, $g$ y $G_R$ son las funciones de Fermi, la degeneración y la función de Green retardada.
\end{frame}

\begin{frame}
  Los elementos de la matriz de transición se relacionan con la velocidad del electrón,
  $$v_{\eta} = \frac{1}{\hbar}\frac{\partial\zeta}{\partial k_{\eta}},$$
  mediante la relación:

  \begin{equation}
    | t_ {\eta, \eta'} |^ {2} = \frac{2 \pi \hbar^{2} }{L_{\eta} L_{\eta'}}v_{\eta} v_{\eta'}.
  \end{equation}
\end{frame}

\begin{frame}
  La conductividad se obtiene de la siguiente manera:
  \begin{equation}
    \sigma_{xx} = \frac{L_x}{L_y}\left(\frac{dI}{dU_g}\right)_{U_g=0},
  \end{equation}

  \begin{equation}
    \sigma_{xx}=\sigma_{o}\sum_{n,k_{y},\beta}\int |t_{x,x}|^{2}\left(-\frac{\partial f}{\partial\epsilon'}\right)\left[2\textbf{Im}G_{n,k}^{R}(\epsilon')\right]^{2}d\epsilon,
    \label{eq:conductividad}
  \end{equation}

  donde se ha tomado $\rho(\epsilon) = - \frac{1}{\pi}\textbf{Im} G_{R} (\epsilon)$,
  $\sigma_{o} = \frac{2 e^{2} n_{L} L_{x}}{\pi \hbar L_{y}}$,
  $ L_{x (y)} $ son las dimensiones de la muestra y $ n_{L} = \Phi / \Phi_{o} $, ($\Phi_{0} = \frac{2\pi}{\hbar})$.
\end{frame}

\subsection{Oscilaciones magnéticas a altas temperaturas}

\begin{frame}
  A partir de la ecuación \ref{eq:conductividad}, se toma:

  \begin{equation}
    [2\textbf{Im}G^{R}(\epsilon)]^2 = G_{R}^{2}+G_{A}^{2}-2G_{R}G_{A},
    \label{2ImGr2}
  \end{equation}
  $G_A$ es la función avanzada de Green que toma la forma \cite{Vega2016}-\cite{Salazar2016}:
  \begin{equation}
    G_A(\epsilon) = \frac{1}{\epsilon'-\epsilon_n-\Sigma^A(\epsilon)} + \frac{1}{\epsilon'+\epsilon_n-\Sigma^A(\epsilon)},
  \end{equation}
\end{frame}

\begin{frame}
  En una forma compacta $G_R$ y $G_A$ pueden escribirse:\\
  \begin{align}
    G_R(\epsilon) = \frac{2(\epsilon'-i |\textbf{Im}\Sigma|)}{(\epsilon'-i |\textbf{Im}\Sigma|)^2-\epsilon_{n}^{2}}    \qquad G_A(\epsilon) = \frac{2(\epsilon'+i |\textbf{Im}\Sigma|)}{(\epsilon'+i |\textbf{Im}\Sigma|)^2-\epsilon_{n}^{2}}.
  \end{align}

  \vspace{0.5cm}
  donde la función $\Sigma^R(\epsilon) = Re\Sigma(\epsilon)+i Im\Sigma(\epsilon)$ es la autoenergía retardada compleja.
\end{frame}

\begin{frame}
  Usando la fórmula de sumatoria de Poisson:
  \begin{equation}
    \sum_{n=-\infty}^{\infty}F(n) = \textbf{Re}\sum_{r=-\infty}^{\infty}\int_{0}^{\infty}F(y)e^{2\pi i r y}dy,
  \label{Fn}
  \end{equation}
  y aplicando el teorema del residuo, donde los polos est\'an dados por $\pm2i\epsilon|Im\Sigma|/\varepsilon_{a}^{2}$, se obtiene:
  \begin{equation}
    \sum_{n=-\infty}^{\infty}G_R G_A =\textbf{Re} \sum_{r=-\infty}^{\infty}\frac{4\pi}{\epsilon_{f}^{2}}\frac{\epsilon'}{|\textbf{Im}\Sigma|} e^{-4\pi r\frac{\epsilon'|\textbf{Im}\Sigma|}{\epsilon_{f}^{2}}} e^{2\pi i r \frac{\epsilon'^2}{\epsilon_{f}^{2}}}.
    \label{GrGa}
  \end{equation}
\end{frame}

\begin{frame}
  Teniendo en cuenta:
  \begin{equation}
    Res\left(h,z_{0}\right)=\lim_{z\rightarrow z_{0}}\frac{1}{\left(n-1\right)!}\frac{d^{n-1}}{dz^{n-1}}\left\{ \left(z-z_{0}\right)^{n}h\left(z\right)\right\}.
    \label{res}
  \end{equation}

  se obtiene:
  \begin{equation}
    G_{R}^{2}=\frac{4\left(\epsilon'^{2}-2i\epsilon'|Im\Sigma|\right)}{\epsilon_{f}^{4}\left(-\frac{2i\epsilon'|Im\Sigma|}{\epsilon_{f}^{2}}-x^{2}\right)^{2}},
    \label{Gr2}
  \end{equation}
\end{frame}

\begin{frame}
  Aplicando la ecuaci\'on (\ref{Fn}) se tiene:
  \begin{equation}
    \sum_{n=-\infty}^{\infty}G_{R}^{2}=\sum_{r=-\infty}^{\infty}\frac{4\left(\epsilon'^{2}-2i\epsilon'|Im\Sigma|\right)}{\epsilon_{f}^{4}}e^{2\pi ir\left(\frac{\epsilon'^{2}}{\epsilon_{f}^{2}}\right)}\int_{-\infty}^{\infty}\frac{e^{2\pi irx}}{\left[-\frac{2i\epsilon'|Im\Sigma|}{\epsilon_{f}^{2}}-x\right]^{2}}dx.
    \label{Gr22}
  \end{equation}
  Aplicando la expresi\'on (\ref{res}) en la ecuaci\'on (\ref{Gr2}) se obtiene
  \begin{equation}
      \sum_{n=-\infty}^{\infty}G_{R}^{2} =\textbf{Re} \sum_{r=-\infty}^{\infty}\frac{4\pi}{\epsilon_{f}^{2}}\frac{\epsilon'}{|\textbf{Im}\Sigma|} e^{2\pi i r \frac{\epsilon'^2}{\epsilon_{f}^{2}}} e^{-4\pi |r|\frac{\epsilon'|\textbf{Im}\Sigma|}{\epsilon_{f}^{2}}} \left( \frac{4\pi |r|\epsilon'|\textbf{Im}\Sigma|}{\epsilon_{f}^{2}}  \right).
      \label{Gr23}
  \end{equation}
\end{frame}

\begin{frame}
  Igualmente se tiene que
  \begin{equation}
    \sum_{N}^{\infty}G_{A}^{2}=\sum_{r=-\infty}^{\infty}\frac{4\pi\epsilon'}{\epsilon_{f}^{2}|Im\Sigma|}e^{2\pi i\mid r\mid\left(\frac{\epsilon'^{2}}{\epsilon_{f}^{2}}\right)}e^{-\frac{4\pi\mid r\mid\epsilon'|Im\Sigma|}{\epsilon_{f}^{2}}}\left[\frac{4\pi\mid r\mid\epsilon'|Im\Sigma|}{\epsilon_{f}^{2}}\right].
    \label{Ga2}
  \end{equation}
  Sustituyendo (\ref{GrGa}), (\ref{Gr23}) y (\ref{Ga2}) en (\ref{2ImGr2}), se obtiene:

  \begin{equation}
    \left[2ImG_{R}\left(k\right)\right]^{2}=\sum_{r=-\infty}^{\infty}\frac{4\pi\epsilon'}{\epsilon_{f}^{2}\mid Im\Sigma\mid}R_{D}e^{2\pi ir\left(\frac{\epsilon'^{2}}{\epsilon_{f}^{2}}\right)}\left[1+\frac{8\pi\mid r\mid\mid Im\Sigma\mid\epsilon'}{\epsilon_{f}^{2}}\right].
    \label{2ImGr22}
  \end{equation}
\end{frame}

\begin{frame}
  Luego, la conductividad longitudinal toma la forma:
  \begin{equation}
    \sigma_{xx} = g \frac{2 e^2 L_x}{\hbar L_y \epsilon_{f}^{2}} \textbf{Re} \sum_{k,r}\int\left| t_{k,k}\right|^2 G_{xx}(\epsilon,r) \left(-\frac{\partial f}{\partial\epsilon}\right)d\epsilon,
    \label{eq:sigmaxx}
  \end{equation}

  donde $$
      G_{xx}(\epsilon,r) = \frac{\epsilon'}{|\textbf{Im}\Sigma|} e^{2\pi i r \frac{\epsilon'^2}{\epsilon_{f}^{2}}} R_D(\epsilon,r) \left[ 1+8\pi |r| \frac{|\textbf{Im}\Sigma|\epsilon'}{\epsilon_{f}^{2}}\right].
    $$
  $$ R_D(\epsilon,r) = \exp\left( -4\pi |r|\frac{|\textbf{Im}\Sigma|\epsilon'}{\epsilon_{f}^{2}}\right) $$
\end{frame}

\begin{frame}
  Introduciendo el factor $S(\lambda,\delta)$ se puede escribir (\ref{eq:sigmaxx}) de forma compacta:
  \begin{equation}
    S(\lambda,\delta) = \sum_{r=-\infty}^{\infty} e^{-|r|\lambda}\cos\delta r = \frac{\sinh\lambda}{\cosh\lambda-\cos\delta},
  \end{equation}
  donde
  \begin{equation}
    \lambda= \frac{4\pi\epsilon'|\textbf{Im}\Sigma|}{\epsilon_{f}^{2}}
  \end{equation}
  y
  \begin{equation}
    \delta= 2\pi\frac{\epsilon'^2}{\epsilon_{f}^{2}}.
  \end{equation}
\end{frame}

\begin{frame}
  La conductividad se puede escribir como $\sigma_{xx} = \sigma_B+ \sigma_Q$ donde:
  \begin{equation}
    \sigma_B= \sigma_{o}\int d\epsilon N(\zeta)\frac{\epsilon}{|\textbf{Im}\Sigma|}\left(-\frac{\partial f}{\partial\epsilon}\right)S(\lambda,\delta).
    \label{eq:sigmaB}
  \end{equation}
  Por otra parte,
  \begin{equation}
    \sigma_Q= -\sigma_{o}\int d\epsilon N(\zeta)\frac{\epsilon}{|\textbf{Im}\Sigma|}\left(-\frac{\partial f}{\partial\epsilon}\right) \left(\lambda_o \frac{\partial S(\lambda,\delta)}{\partial \lambda}\right),
    \label{eq:sigmaQ}
  \end{equation}
  donde $\lambda_o=8 \pi\epsilon'|\textbf{Im}\Sigma|/\epsilon_{f}^{2}$.
\end{frame}

\begin{frame}
  El parámetro $N(\zeta)$ está relacionado con los estados de Bloch,
  \begin{equation}
    N(\zeta)=\frac{L_{x}L_{y}}{a_{x}\pi^{2} \varepsilon_{f}^{2}}\int_{0}^{a_{x}/2l^{2}}dk_{y} \int_{0}^{2\pi}d\beta|t_{x,x}|^{2}.
  \end{equation}

  Luego, la conductividad longitudinal puede escribirse como:
  \begin{equation}
    \sigma_{xx}=\sigma_{o}\int  N(\zeta)\left(-\frac{\partial f}{\partial\epsilon}\right)\frac{\epsilon}{\mid \textbf{Im}\Sigma\mid}G_{xx}(\delta,\lambda)d\epsilon,
    \label{condxx}
  \end{equation}

  donde $G_{xx}(\delta,\lambda)=\left[ 1-\lambda_{o}\frac{\partial }{\partial\lambda}\right] S(\delta,\lambda)$.
\end{frame}

\begin{frame}
  \begin{figure}[h!]
    \begin{center}
      \includegraphics[scale=0.5]{graficas/fig5.png}
      \caption{$\sigma_{xx}$ vs. $B$ para $T = 150 K$ (negra), $T = 200 K$ (roja) y $T = 250 K$ (azul).}
      \label{sigmavsB}
      \centering
    \end{center}
  \end{figure}
\end{frame}

\begin{frame}
  \begin{figure}[h!]
    \begin{center}
      \includegraphics[scale=0.5]{graficas/fig6.png}
      \caption{$\sigma_{xx}$ vs. $B$ ($T = 300 K$).}
      \label{sigma300k}
      \centering
    \end{center}
  \end{figure}
\end{frame}

\begin{frame}
  \begin{figure}[h!]
    \begin{center}
      \includegraphics[scale=0.5]{graficas/fig8.png}
      \caption{$\sigma_{B}$ vs. $B$ $T = 150 K$ (negra), $T = 200 K$ (roja) y $T = 250 K$ (azul).}
      \label{cont_Boltz}
      \centering
    \end{center}
  \end{figure}
\end{frame}

\begin{frame}
  \begin{figure}[h!]
    \begin{center}
      \includegraphics[scale=0.5]{graficas/fig9.png}
      \caption{$\sigma_{Q}$ vs. $B$ $T = 150 K$ (negra), $T = 200 K$ (roja) y $T = 250 K$ (azul).}
      \label{cont_Cuant}
      \centering
    \end{center}
  \end{figure}
\end{frame}

\begin{frame}
  \begin{figure}[h!]
    \begin{center}
      \includegraphics[scale=0.4]{graficas/fig7.png}
      \caption{$\sigma_{xx}$, $\sigma_{B}$ y $\sigma_{Q}$ vs. $B$ ($T = 300 K$).}
      \label{contrib}
      \centering
    \end{center}
  \end{figure}
\end{frame}

\begin{frame}
  Para todos los cálculos, los parámetros utilizados son $v_{f}=1.15\times 10^{6}$ m/s, $a_{x}=a_{y}=a=50$ nm,
  $V_{o}=1\times10^{-6}$ eV, $U_{x}=U_{y}=U_{o}=0.5\times10^{-3}$ eV, $n_{i}=3\times10^{15}$ m$^{-2}$
  y la energía de Fermi $\mu_{o}=100$ meV.\\

  \vspace{0.5cm}

  Las oscilaciones a altas temperaturas debidas a la contribución $\sigma_{Q} $ responden a variaciones periódicas
  del parámetro de desorden, las cuales están determinadas por la variación del factor de forma $ S(\delta,\lambda)$.
\end{frame}

\subsection{Tiempo de relajación del electrón}

\begin{frame}
\end{frame}

\begin{frame}
\end{frame}

\begin{frame}
\end{frame}

\begin{frame}
\end{frame}

\begin{frame}
\end{frame}
