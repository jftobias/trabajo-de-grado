\section{Introducción}

\begin{frame}{Grafeno}
	\justifying
	\begin{multicols}{2}

		\underline{Grafeno}:\\
		Es una de las formas alotrópicas del carbono, entre las que se encuentran el grafito, diamante, entre otras. En la cual los átomos están organizados en un arrreglo bidimensional y con un patrón hexagonal en forma de panel de abeja.

		\begin{figure}
		\includegraphics[width=5.5cm]{graficas/alotropo.jpg}
		\caption{ Alótropos del carbono. \textit{Tomado de }\cite{Neto2006}}
		\label{grafeno}
		\end{figure}
	\end{multicols}
\end{frame}

\begin{frame}
	\justifying
	\begin{multicols}{2}
		Los electrones en el grafeno se comportan como fermiones de Dirac sin masa \cite{Katsnelson2007}. A temperatura ambiente, los electrones pueden viajar varias micras sin presentar dispersión. Es buen conductor, resistente y transparente.
		\begin{figure}
			\includegraphics[width=5cm]{graficas/propiedadesGrafeno.jpg}
			\caption{Propiedades del grafeno}
		\end{figure}
		\end{multicols}
\end{frame}

\begin{frame}
	\justifying
	Las oscilaciones magnéticas son un fenómeno conocido en la física de la materia condensada, generalmente observado a bajas temperaturas.
	En el grafeno, persiste incluso a temperaturas ambiente. Esto se atribuye a la periodicidad de las superredes de grafeno\cite{Kumar2017}.

	\begin{figure}
		\subfigure[Superred de grafeno]{\includegraphics[width=4cm]{graficas/oscilaciones_superred.jpg}}
		\subfigure[grafeno]{\includegraphics[width=4cm]{graficas/oscilaciones_grafeno.jpg}}
	\caption{$\rho$ vs $B$ a diferentes temperaturas. \textit{Tomadas de \cite{Kumar2017}}}
	\end{figure}
\end{frame}

\begin{frame}
	\justifying
	\begin{multicols}{2}
		Las oscilaciones cuánticas involucran trayectorias ciclotrónicas \cite{Chen2014}. Lo que conlleva a la cuantización de Landau(L.L.) y a oscilaciones de Shubnikov-de Hass (SdH) \cite{Fujita2014}.
		Incluso en el grafeno, las oscilaciones SdH raramente persisten a temperaturas superiores a los 100K \cite{Kishigi2014}, se necesitan campos magnéticos grandes para observarlas \cite{Novoselov2007}
	\end{multicols}
\end{frame}

\begin{frame}
	\justifying
	\begin{multicols}{2}
		Los sistemas electrónicos tambien presentan oscilaciones magnéticas,
		referidos a las mariposas de Hofstadter (HB) \cite{Yu2014}-\cite{Yang2016}.\\
		\vspace{0.5cm}
		Estudios de transporte eléctrico en superredes de grafeno sobre h-BN \cite{Yankowitz2012}
		muestran características de las HB originadas en la superred en un campo magnético \cite{BenShalom2016}.
		\begin{figure}
			\includegraphics[width=6cm]{graficas/heterostructures.jpg}
			\caption{Superredes de grafeno. \textit{Tomado de} \cite{Geim2013}}
			\label{heterostructures}
		\end{figure}
	\end{multicols}
\end{frame}

\begin{frame}
	\justifying
	El estudio de estas oscilaciones magnéticas a altas temperaturas en sistemas basados
	en grafeno debido a la periodicidad de estados de Bloch en campos magnéticos,
	llamadas oscilaciones de Brown-Zak (BZ) \cite{Kumar2017}. \\
	\vspace{0.5cm}
	En este contexto, el presente trabajo pretende ahondar en el estudio de las propiedades magneto-oscilatorias
	de las estructuras basadas en grafeno en presencia de un campo magnético a altas temperaturas.

\end{frame}
